% This template was originally by R. Jacob Vogelstein
% Updated on March 1, 2010 by Noah J. Cowan
% Updated on August 30, 2016 by Kelly C. Chang


\documentclass[12pt,oneside,final]{thesis}

\usepackage{threeparttable}
\usepackage{cite}
\usepackage{amsmath,amsfonts}
\usepackage{graphicx}
\graphicspath{{./figs/}}
\usepackage{array}
% wrapfig is fragile: use sparingly
\usepackage{wrapfig} 
%\usepackage{times}  % Use this for ugly fonts
\usepackage{textcomp}
\usepackage[bottom]{footmisc}
\usepackage{afterpage}
\usepackage[titletoc]{appendix}
\usepackage{bibentry}
\nobibliography*

\makeatletter
\def\bstctlcite{\@ifnextchar[{\@bstctlcite}{\@bstctlcite[@auxout]}}
\def\@bstctlcite[#1]#2{\@bsphack
  \@for\@citeb:=#2\do{%
    \edef\@citeb{\expandafter\@firstofone\@citeb}%
    \if@filesw\immediate\write\csname #1\endcsname{\string\citation{\@citeb}}\fi}%
  \@esphack}
\makeatother

\usepackage[section]{placeins}

\usepackage{upgreek}
\usepackage[hyperfootnotes=false,hidelinks]{hyperref}
\usepackage{setspace}
\usepackage[nopostdot,nonumberlist,nogroupskip,style=super,acronym,toc,ucmark]{glossaries}

\let\thempfootnote\thefootnote

\usepackage{booktabs}
\usepackage{multirow}
\usepackage{longtable}
\usepackage[margin=10pt,labelfont=bf]{caption}
\captionsetup[table]{belowskip=12pt, aboveskip=0pt}
\captionsetup[figure]{belowskip=12pt, aboveskip=4pt}
% the following format will be used to emulate the captions produced by fltpage
\DeclareCaptionFormat{cont-page}{#1 #2 #3 \hrulefill\\\hspace*{\fill}\emph{(next page)}}
\DeclareCaptionLabelFormat{adja-page}{\hrulefill\\#1 #2 \emph{(previous page)}}
%\usepackage{CV}

\usepackage{enumitem}
\newlist{inlinelist}{enumerate*}{1}
\setlist*[inlinelist,1]{%
  label=(\arabic*),
}

\usepackage{fancyhdr}    % Use nice looking headers along with the required footer page numbers   
%\usepackage[hypertex]{hyperref}

%Define the header/footer style
\pagestyle{fancy}
\fancyhf{}
\setlength{\headheight}{15pt}
\lhead{\leftmark}
\cfoot{\thepage}
\renewcommand{\headrulewidth}{0pt}
\fancypagestyle{plain}{% Redefine ``plain'' style for chapter boundaries
\fancyhf{} % clear all header and footer fields
\fancyfoot[C]{\thepage} % except the center
\renewcommand{\headrulewidth}{0pt}
\renewcommand{\footrulewidth}{0pt}}

%\tolerance=10000

\makeglossaries % enable the glossary
\setlength{\glsdescwidth}{\textwidth}
\loadglsentries{glossary}

\usepackage{mathtools}
\DeclarePairedDelimiter\abs{\lvert}{\rvert}%
\DeclarePairedDelimiter\norm{\lVert}{\rVert}%

\newcommand{\expnumber}[2]{{#1}\mathrm{e}{#2}}

\begin{document}
\bstctlcite{BSTcontrol}
\glsdisablehyper

\title{JHU THESIS TEMPLATE}
\author{R. Jacob Vogelstein}
\degreemonth{May}
\degreeyear{2007} 
\dissertation
\doctorphilosophy
\copyrightnotice


% add your chapters, best way is to have separate TeX files for each chapter
\include{chapter0}
\include{chapter1}

\begin{appendices}
%\appendix{}
%\label{sec:appendix}
%
\chapter[]{Chapter}

This is a chapter.

\section{Section}
\label{sec:asection}

This is a section.  Here's a reference to a different section:
\ref{sec:asubsection}.

\subsection{Subsection}
\label{sec:asubsection}

This is a subsection.\footnote{footnote text}


\end{appendices}

%% REFERENCES

% if you use BIBTEX
\bibliographystyle{IEEEtran}
\bibliography{thesis.bib}

% generate list of abbreviations
\listofabbrevs

% list of symbols
\listofsymbols

\begin{vita}

\begin{wrapfigure}{l}{0pt}
\includegraphics[height=2.4in,clip,keepaspectratio]{rjvheadshot}
\end{wrapfigure}

R.\ Jacob Vogelstein received the Sc.\ B.\ degree in Bio-Electrical Engineering 
from Brown University in 2000,  and enrolled in the Biomedical Engineering 
Ph.D.\ program at Johns Hopkins University in 2001.  He was inducted into the 
Tau Beta Pi and Sigma Xi honor societies in 1999, won the Brown University 
Engineering Department's Outstanding Student Award in 2000, and received a 
National Science Foundation Graduate Research Fellowship in 2002.  His research 
focuses on neuromorphic and neuroprosthetic devices, and his papers have been 
finalists in the student paper competitions at the 2004 IEEE International 
Conference of the Engineering in Medicine and Biology Society and the 2004 IEEE 
International Conference on Electronics, Circuits and Systems.

Starting in June 2007, Jacob will work on the ``Revolutionizing Prosthetics 2009'' 
project at the Johns Hopkins University Applied Physics Laboratory in Laurel,
MD, where he will help to create the next-generation of upper-arm
neuroprostheses.  

\end{vita}
\end{document}
